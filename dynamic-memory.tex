\section{Alocação de Memória Dinâmica}
Nós comentamos um pouco na primeira seção sobre alocação estática de memória. Agora vamos comentar sobre
\textbf{Alocação Dinâmica de Memória}. Nesse caso, a alocação ocorre durante o \textit{runtime} do programa.
Mas, porque precisamos disso? Alocação de memória durante o tempo de execução do programa nos permite uma maior
flexibilidade com a memória, ou seja, conseguimos gerenciar a memória que precisamos, sem alocar mais
ou menos que o necessário. Na maioria das vezes, vamos usar alocação dinâmica para tipos de dados
derivados, ou seja, \textbf{\textit{pointer types}}, \textbf{\textit{array}}, \textbf{\textit{structures}} (registros), etc.

\subsection{Como alocar dinamicamente?}
As 4 funções em C referente a alocação dinâmica de memória estão definidas na \textit{stdlib.h}, sendo elas:
\begin{itemize}
  \item \textit{\textbf{malloc()} \footnote{Mais comum quando trabalhamos com alocação dinâmica.}}: abreviação para \textit{memory allocation}. Essa função reserva um bloco de memória
    com o tamanho especificado de \textit{bytes}, e retorna um \textbf{ponteiro} do tipo \textbf{\textit{void}}
    que podemos dar \textit{cast} para outro tipo de dado. 
  \item \textbf{\textit{realloc()}}: caso a memória previamente alocada dinamicamente for insuficiente (ou maior do que necessário)
    podemos mudar o tamanho dela usando a função \textbf{\textit{realloc}}, especificando o novo tamanho da memória.
  \item \textbf{\textit{calloc()}}: abreviação para \textit{contiguous allocation}. A diferença entre \textbf{\textit{malloc}} e \textbf{\textit{calloc}}
    é que o segundo aloca a memória e inicializa todos os \textit{bits} para zero.
  \item \textbf{\textit{free()}}: memória dinamicamente alocada não são liberadas sozinhas, já que as mesma
    estão sendo alocadas na Heap. Dessa forma, precisamos explicitamente liberar elas usando a função \textbf{\textit{free()}}.
    Caso não liberamos a memória, pode ocorrer o \textbf{\textit{Memory Leak}}, reduzindo a performance e a
    quantidade de memória disponível.
\end{itemize}

\subsection{Sintaxe das Funções}
\begin{lstlisting}[language=C]
ptr = (castType*)malloc(size)
ptr = (castType*)calloc(n, size)
ptr = realloc(ptr, x)
free(ptr)
\end{lstlisting}

\subsection{Exemplo de Uso - \textit{malloc()}}
Fazemos parte de uma universidade e precisamos guardar alguns dados dos alunos:
\begin{itemize}
    \item Nome Completo;
    \item Data de Nascimento;
    \item RA;
    \item Quantidade de matérias cursadas;
    \item Matérias cursadas;
    \item Data limite para Integralização;
\end{itemize}

Todos os dados mencionados acima serão do tipo \textbf{\textit{string}} ou um \textbf{\textit{array}} de \textbf{\textit{strings}}. Sabendo disso, como podemos abordar esse problema?

\subsubsection{Criando a Estrutura Aluno}
Como estamos na linguagem C, podemos usar uma \textbf{\textit{struct} Aluno} a qual é composta pelos dados acima. Dessa forma, teremos:
\begin{lstlisting}[language=C]
typedef struct Aluno {
    char *nomeCompleto;
    char *dataDeNascimento;
    char *ra;
    char **materiasCursadas;
    char *limiteIntegralizacao;
    int qtdMaterias;
} Aluno;
\end{lstlisting}

\subsubsection{Função para Criar um Novo Aluno}
Até momento apenas definimos a estrutura aluno. Não podemos de fato usar ainda. Lembre-se, aqui estamos tratando de uma \textbf{\textit{struct}}, logo temos que alocar espaço na memória para estarmos manipulando a mesma. Para isso, podemos fazer uma função \textbf{newAluno()}, que recebem alguns parâmetros e retorna um \textbf{ponteiro} para a \textbf{\textit{struct}} do tipo \textbf{\textit{Aluno}}:
\begin{lstlisting}[language=C]
Aluno *newAluno(char *nomeCompleto, char *dataDeNascimento, char *ra, char *limiteIntegralizacao)
{
    Aluno *aux = (Aluno *)malloc(sizeof(Aluno);
    aux->nomeCompleto = nomeCompleto;
    aux->dataDeNascimento = dataDeNascimento;
    aux->ra = ra;
    aux->materiasCursadas = (char **)malloc(sizeof(char *) * 7);
    aux->limiteIntegralizacao = limiteIntegralizacao;
    aux->qtdMaterias = 0;
    return aux;
}
\end{lstlisting}

Note que para as materias cursadas, tivemos um \textbf{\textit{cast}} para ponteiro de ponteiro de \textbf{\textit{char}} e no \textbf{\textit{malloc()}}, usamos o \textbf{\textit{sizeof(char *)}} a multiplicamos esse valor por 7 \footnote{O valor 7 foi escolhido aleatoriamente. Podemos colocar qualquer valor aqui.}, assim podemos ter 7 registros. Percebe um padrão? Apenas removemos um '*'. No caso de \textbf{\textit{char ***}}, teríamos \textbf{\textit{sizeof(char **)}}.

\subsubsection{Adicionando Matérias}
Agora, temos que pensar em como podemos adicionar matérias. No nosso caso, \textbf{\textit{materiasCursadas}} é um \textbf{\textit{char **}}, ou seja, ponteiro de ponteiro. Dessa forma, temos um \textbf{\textit{array}} de \textbf{\textit{strings}}. Sabendo disso, podemos criar uma função \textbf{\textit{addMaterial()}}, que recebe como parâmetro um ponteiro para Aluno, e um \textbf{\textit{string}} matéria:
\begin{lstlisting}[language=C]
void addMaterial(Aluno *aluno, char *materia)
{
    printf("Adicionando a materia: %s\n", materia);
    aluno->materiasCursadas[aluno->qtdMaterias++] = materia;
}
\end{lstlisting}

No bloco de código acima, estamos pegando o vetor \textbf{\textit{materiasCursadas}} e, baseando na quantidade de matérias que temos (inicialmente 0), estamos adicionando uma nova matéria e, em seguida, temos que somar 1 à \textbf{\textit{qtdMateria}}, já que uma nova matéria foi adicionada.

\subsubsection{Usando Aluno}
Agora que temos as principais parte do código prontas, podemos estar usand-o. Primeiramente, vamos criar um novo aluno passando alguns argumentos para ele e, em seguida, vamos adicionar algumas matérias:
\begin{lstlisting}[language=C]
int main()
{
    // Criando o aluno "Miguel"
    Aluno *aluno = newAluno("Miguel", "24/05/2000", "567784", "31/12/2027");
    
    // Adicionando materias para Miguel
    addMateria(aluno, "SI201");
    addMateria(aluno, "SI401");
    addMateria(aluno, "TT002");
    addMateria(aluno, "ST562");
    
    // Printando materias de Miguel
    for(int i = 0; i < aluno->qtdMaterias; i++)
    {
        printf("%s\n", aluno->materiasCursadas[i]);
    }

    return 0;
}
\end{lstlisting}

\subsubsection{Pontos Importantes}
Apesar do código acima estar funcional, tem algumas coisas que podemos melhorar:
\begin{itemize}
    \item Podemos criar uma função para printarmos as informações dos alunos;
    \item Quando alocamos \textbf{\textit{materiasCursadas}}, pegamos o tamanhao de \textbf{\textit{char *}} e multiplicamos por 7, ou seja, podemos ter 7 entradas dentro desse nosso \textbf{\textit{array}}. O que acontece se tivermos mais? E se tivermos menos? Como podemos estar contornando isso? (Dica: função relacionada a alocação dinâmica);
    \item Como podemos fazer para excluirmos uma matéria?
\end{itemize}